\documentclass{article}
\usepackage[utf8]{inputenc}
\usepackage{hyperref}

\title{Using Maimonides' rule to estimate the effect of clas size on student achievement}
\author{Joshua D Angrist, Victor Lavy}
\date{Published in: Quarterly Journal of Economics}

\begin{document}

\maketitle

\section*{Source}
https://economics.mit.edu/sites/default/files/publications/Using%20Maimonides%20Rule%20To%20Estimate%20the%20Effect%20of%20Cl.pdf

\section*{Objective}
This study aims to utilize the inference method of Regression Discontinuity Design (RDD) to pose a causal relationship between decreased class size and increased student achievement in classrooms of Fourth and Fifth graders.

\section*{Methodology}
Using data from Israeli Fourth and Fifth grade classrooms and employing a cutoff at a class size of 40, the research establishes the causal relationship posed in the objective. In this study, the RDD methodology resembles a fuzzy RDD approach.

\section*{Reason}
This study aims to provide insight on a topic that is prevalent to billions of children all over the world. Many students cannot function well in large classes and would be more productive in an environment with fewer students.

\section*{Data}
The study compiles data from Israeli Fourth and Fifth grade classrooms using including class sizes and test scores of both Math and Reading.

\section*{Results}
The findings reveal a significant impact of decreased class size on increased student achievement. The study notes that although the data is from Israel, the results are likely to be true for other developed countries as well.

\end{document}
